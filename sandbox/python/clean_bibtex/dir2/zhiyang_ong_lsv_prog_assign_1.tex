\documentclass[letter,12pt]{article}
%%%%%%%%%%%%%%%%%%%%%%%%%%%%%%%%%%%%%%%%%%%%%%%%%
%	\usepackage{graphicx}
%	\usepackage{amsmath}
%	\usepackage{array}
%	\usepackage{amssymb}
%	\usepackage{setspace}
%	%\usepackage[margin=1.5cm,vmargin={0pt,1cm},nohead]{geometry}
%	\usepackage[margin=1in,vmargin={1in,1in}]{geometry}
%	% Package that has the symbol for ``:=''
%	\usepackage{txfonts}
%	% Create fancy headers and footers for this document
%	\usepackage{fancyhdr}
%	%\usepackage{cite}
%	% The ``cite'' package causes the hyperlinks for the in-text references/citations to fail. I believe it is because this package overrides the default package for referencing. Hence, only use the ``cite'' package with the IEEE format.
%	% Package for ``turnstile'' binary relations, where letters are defined above and below symbols
%	\usepackage{turnstile}
%	\usepackage{extarrows}
%	% Package that provides the cross symbol
%	\usepackage{ifsym}
%	\usepackage{marvosym}
%	% Commands for using the package for hyperlinks - 
%	\usepackage[pdftex,
%		pdftitle={Graphics and Color with LaTeX},
%		pdfauthor={Patrick W Daly},
%		pdfsubject={Importing images and use of color in LaTeX},
%		pdfkeywords={LaTeX, graphics, color},
%		pdfpagemode=UseOutlines,bookmarks, bookmarksopen,
%		pdfstartview=FitH, colorlinks, linkcolor=blue, citecolor=blue, urlcolor=red,
%	]{hyperref}
%	\hypersetup{colorlinks, linkcolor=blue}
%	% Concatenate references
%	\usepackage{cite}


%	% Package for tyepsetting algorithms and heuristics
%	\usepackage{listings}
%	\lstset{language=[GNU]C++}

%%%%%%%%%%%%%%%%%%%%%%%%%%%%%%%%%%%%%%%%%%%%%
%	Additional packages
\input{/data/others/grappanotes/others/preamble}
%	AMS theorem package
\usepackage{amsthm}




% definition of new \LaTeX command for the citation: \cite{Cimatti08} and \cite{Barrett09}
% This allows mathematical/logic symbols to be typeset with the font ``Zapf Chancery'' in ``\LaTeX\ math mode''. To typeset symbols in such font, try: \mathpzc{ABCdef123}
\DeclareMathAlphabet{\mathpzc}{OT1}{pzc}{m}{it}

%%%%%%%%%%%%%%%%%%%%%%%%%%%%%%%%%%%%%%%%%%%%%
% Start of document
\begin{document}
\title{Logic Synthesis and Verification \\
Programming Assignment \#1}
%{\Large Programming Assignment \#1}}
\date{\today}
\author{Zhiyang Ong
	\thanks{Email correspondence to: \href{mailto:ongz@acm.org}{ongz@acm.org}}
}
\maketitle






%%%%%%%%%%%%%%%%%%%%%%%%%%%%%%%%%%%%%%%%%%%
\section*{Declaration}
\label{sec:declaration}

I did this assignment on my own without any collaborators. %The mathematical programming package that I have used is \cite{Makhorin2012}. 










%%%%%%%%%%%%%%%%%%%%%%%%%%%%%%%%%%%%%%%%%%%
\section{Introduction}
\label{sec:intro}

The resources for {\it ABC} that I have used for my programming assignment \#1 are: \cite{Tu2013,BLSVG2011,Mishchenko20XY}.




%%%%%%%%%%%%%%%%%%%%%%%%%%%%%%%%%%%%%%%%%%%
\section{Question 1: Using ABC}
\label{sec:usingabc}

Since the question did not state if the magnitude comparator is for signed or unsigned numbers, I assume that the comparison is for unsigned numbers, $A$ (see Equation (\ref{eqn:comparatorinputA})) and $B$ (see Equation (\ref{eqn:comparatorinputB})). This comparator has the following inputs:
\begin{equation}
\label{eqn:comparatorinputA}
A = A_{3}A_{2}A_{1}A_{0}
\end{equation}
\begin{equation}
\label{eqn:comparatorinputB}
B = B_{3}B_{2}B_{1}B_{0}
\end{equation}


If $A > B$, the magnitude comparator should output a one (or logic high signal) \cite[\S4.8, pages 148--150]{Mano2013} \cite[\S5.10, pages 193--196]{Gajski1997} \cite[\S3.1.2, pages 120--121]{Katz1994}. The comparison is carried out by comparing the relative magnitudes of the $i^{th}$ digit from $A$ and $B$, $\forall i \in \{3, 2, 1, 0\}$, from the most to least significant position. The comparisons terminate when the $i^{th}$ comparison returns an unequal pair of digits (i.e., $A_{i} \neq B_{i}$), or when no more comparisons can be made (i.e., comparisons of all four pairs of digits are made). For $A_{i} \neq B_{i}$, if $A = 1$ and $ B = 0$, $A > B$ (i.e., output is a logic `1'). Else, output value is `0'. This is expressed in Equation (\ref{eqn:comparatorcomparison}).

\begin{equation}
\label{eqn:comparatorcomparison}
A > B = A_{3}B^{\prime}_{3} + x_{3}A_{2}B^{\prime}_{2} + x_{3}x_{2}A_{1}B^{\prime}_{1} + x_{3}x_{2}x_{1}A_{0}B^{\prime}_{0}
\end{equation}

The $x_{i}$s in in Equation (\ref{eqn:comparatorcomparison}) can be calculated as follows:
\begin{equation}
\label{eqn:comparatorxi}
x_{i} = A_{i}B_{i} + A^{\prime}_{i}B^{\prime}_{i}, \forall i \in \{0, 1, 2, 3\}
\end{equation}


\cite[\S3.1.2, pages 113--115]{Ashenden2008a}



\begin{figure}
\centering 
%\includegraphics[width=6in]{comparator.pdf}
\includegraphics[width=6in]{./pictures/comparator}
\caption{4-bit magnitude comparator \cite[\S4.8, page 150]{Mano2013}, with outputs for the comparisons of $A < B$, $A > B$, and $A = B$}
\label{fig:INSERT LABEL OF FIGURE}
\end{figure}



%%%%%%%%%%%%%%%%%%%%%%%%%%%%%%%%%%%%%%%%%%%
\section{Question 2: ABC Boolean Function Representation}
\label{sec:abcbooleanfnrep}




%%%%%%%%%%%%%%%%%%%%%%%%%%%%%%%%%%%%%%%%%%%
\section{Question 3: Programming ABC}
\label{sec:programmingabc}








%%%%%%%%%%%%%%%%%%%%%%%%%%%%%%%%%%%%%%%%%%%%%
{\linespread{1}
\bibliographystyle{plain}
%\bibliography{/data/research/antipastobibtex/references}
\bibliography{assgin1_references}
}
%%%%%%%%%%%%%%%%%%%%%%%%%%%%%%%%%%%%%%%%%%%%%
\end{document}