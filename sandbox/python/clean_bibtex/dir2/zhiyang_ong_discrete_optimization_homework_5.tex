\documentclass[letter,12pt]{article}
%%%%%%%%%%%%%%%%%%%%%%%%%%%%%%%%%%%%%%%%%%%%%%%%%
%	\usepackage{graphicx}
%	\usepackage{amsmath}
%	\usepackage{array}
%	\usepackage{amssymb}
%	\usepackage{setspace}
%	%\usepackage[margin=1.5cm,vmargin={0pt,1cm},nohead]{geometry}
%	\usepackage[margin=1in,vmargin={1in,1in}]{geometry}
%	% Package that has the symbol for ``:=''
%	\usepackage{txfonts}
%	% Create fancy headers and footers for this document
%	\usepackage{fancyhdr}
%	%\usepackage{cite}
%	% The ``cite'' package causes the hyperlinks for the in-text references/citations to fail. I believe it is because this package overrides the default package for referencing. Hence, only use the ``cite'' package with the IEEE format.
%	% Package for ``turnstile'' binary relations, where letters are defined above and below symbols
%	\usepackage{turnstile}
%	\usepackage{extarrows}
%	% Package that provides the cross symbol
%	\usepackage{ifsym}
%	\usepackage{marvosym}
%	% Commands for using the package for hyperlinks - 
%	\usepackage[pdftex,
%		pdftitle={Graphics and Color with LaTeX},
%		pdfauthor={Patrick W Daly},
%		pdfsubject={Importing images and use of color in LaTeX},
%		pdfkeywords={LaTeX, graphics, color},
%		pdfpagemode=UseOutlines,bookmarks, bookmarksopen,
%		pdfstartview=FitH, colorlinks, linkcolor=blue, citecolor=blue, urlcolor=red,
%	]{hyperref}
%	\hypersetup{colorlinks, linkcolor=blue}
%	% Concatenate references
%	\usepackage{cite}


%	% Package for tyepsetting algorithms and heuristics
%	\usepackage{listings}
%	\lstset{language=[GNU]C++}

%%%%%%%%%%%%%%%%%%%%%%%%%%%%%%%%%%%%%%%%%%%%%
%	Additional packages
\input{/data/others/grappanotes/others/preamble}
%	AMS theorem package
\usepackage{amsthm}




% definition of new \LaTeX command for the citation: \cite{Cimatti08} and \cite{Barrett09}
% This allows mathematical/logic symbols to be typeset with the font ``Zapf Chancery'' in ``\LaTeX\ math mode''. To typeset symbols in such font, try: \mathpzc{ABCdef123}
\DeclareMathAlphabet{\mathpzc}{OT1}{pzc}{m}{it}

%%%%%%%%%%%%%%%%%%%%%%%%%%%%%%%%%%%%%%%%%%%%%
% Start of document
\begin{document}
\title{Discrete Optimization Homework \#5}
\date{\today}
\author{Zhiyang Ong
	\thanks{Email correspondence to: \href{mailto:ongz@acm.org}{ongz@acm.org}}
}
\maketitle






%%%%%%%%%%%%%%%%%%%%%%%%%%%%%%%%%%%%%%%%%%%
\section*{Declaration}
\label{sec:declaration}

I did this assignment on my own without any collaborators. The mathematical programming package that I have used is \cite{Makhorin2012}. 







%%%%%%%%%%%%%%%%%%%%%%%%%%%%%%%%%%%%%%%%%%%
\section{Total Unimodularity}
\label{sec:totalunimodularity}

%%%%%%%%%%%%%%%%%%%%%%%%%%%%%%%%%%%%%%%%%%%
\subsection{Total Unimodular Transportation Problem}
\label{ssec:totalunimodularitytransportationproblem}

Question 1a):\\
\ \\
A transportation problem is defined mathematically as follows \cite[\S8.4, pp.439--447]{Miller2000}:
\begin{eqnarray}
\label{eqn:transportationproblem}
	\begin{gathered}[b]
	{\rm minimize}\ T = \displaystyle\sum^{m}_{i = 1} \displaystyle\sum^{n}_{j = 1} t_{ij}x_{ij} \\
	\underline{x} \in S \\
	{\rm subject\ to:} \\
	\displaystyle\sum^{n}_{j = 1} x_{ij} = a_{i} \\
	\displaystyle\sum^{m}_{i = 1} x_{ij} = b_{j} \\
	x_{ij} \geq 0
	\end{gathered}
\end{eqnarray}

where $m$ is the number of places of origin for the supplies (i.e., goods produced), $n$ is the number of destinations for the supplies, $x_{ij}$ (the decision variable) is the quantity of goods send from a place of origin $i$ to its destination $j$, $t_{ij}$ is the cost of transporting each unit of supply/goods from origin $i$ to destination $j$, $a_{i}$ is the known supply at origin $i \in \{1, 2, \dots, m\}$, and $b_{j}$ is the known demand at destination $j \in \{1, 2, \dots, n\}$. \\

Here, this formulation of the transportation problem is a class of linear programming (LP) optimization problems. The $x_{ij}$'s in the constraints (as shown in Equation \ref{eqn:transportationproblem}) have coefficients that are either 0 or 1. That is, $x_{ij} = 0$ denotes that no goods are send from origin $i$ to destination $j$, and $x_{ij} = 1$ denotes that goods are send from origin $i$ to destination $j$. \\

Consequently, the equations in Equation \ref{eqn:transportationproblem} that represent constraints of the transportation problem can be expressed in the $\mathbf{Ax = B}$ form. The rank of this matrix $\mathbf{A}$, rank($\mathbf{A}$), is determined by the number of linear independent rows/columns in $\mathbf{A}$. However, if $\mathbf{A}$ has linearly dependent rows (or columns), its determinant would be zero. This is because the determinant of a matrix that is composed of linearly dependent vectors is zero \cite[\S5.3, Example 4]{Anton2005} \cite[\S3.2, pp. 171]{Lay2012} \cite[\S1.9.1, pp. 24]{Shilov1977} \cite[\S4.2, pp. 229]{Strang2006}.

An example of a transportation problem is given as follows \cite[\S8.4, pp.443--446]{Miller2000}:
\begin{eqnarray}
\label{eqn:transportationproblemexample}
	\begin{gathered}[b]
	{\rm minimize}\ T = 30x_{11} + 40x_{12} + 20x_{13} + 40x_{21} + 10x_{22} + 60x_{23} \\
	\underline{x} \in S \\
	{\rm subject\ to:} \\
	x_{11} + x_{12} + x_{13} = 10; \\
	x_{21} + x_{22} + x_{23} = 15; \\
	x_{11} + x_{21} = 8; \\
	x_{12} + x_{22} = 12; \\
	x_{13} + x_{23} = 5; \\
	x_{ij} \geq 0
	\end{gathered}
\end{eqnarray}

This can be expressed as a maximization problem of $-T$; i.e., maximize $-T$ \cite[\S22.2, pp. 957]{Kreyszig2011}\cite[\S4.3, pp. 146]{Boyd2004}. The transportation problem in Equation \ref{eqn:transportationproblemexample} can be expressed in the $\mathbf{Ax = B}$ standard form \cite[\S2.1, pp.27--28]{Papadimitriou1998} \cite[\S4.3, pp. 146--147]{Boyd2004} \cite[\S11.2.1, pp. 322--323]{Antoniou2007}, which is indicated as follows.

\begin{eqnarray}
\label{eqn:transportationproblemexampleaxeqb}
\left[
\begin{array}{cccccc}
1 & 1 & 1 & 0 & 0 & 0 \\
0 & 0 & 0 & 1 & 1 & 1 \\
1 & 0 & 0 & 1 & 0 & 0 \\
0 & 1 & 0 & 0 & 1 & 0 \\
0 & 0 & 1 & 0 & 0 & 1
\end{array}
\right]
\left[
\begin{array}{cccccc}
x_{11} \\
x_{12} \\
x_{13} \\
x_{21} \\
x_{22} \\
x_{23}
\end{array}
\right]
=
\left[
\begin{array}{cccccc}
10 \\
15 \\
8 \\
12 \\
5
\end{array}
\right]
\end{eqnarray}

Here, we have:
\[
\mathbf{A} = 
\left[
\begin{array}{cccccc}
1 & 1 & 1 & 0 & 0 & 0 \\
0 & 0 & 0 & 1 & 1 & 1 \\
1 & 0 & 0 & 1 & 0 & 0 \\
0 & 1 & 0 & 0 & 1 & 0 \\
0 & 0 & 1 & 0 & 0 & 1
\end{array}
\right],
\mathbf{x} = 
\left[
\begin{array}{cccccc}
x_{11} \\
x_{12} \\
x_{13} \\
x_{21} \\
x_{22} \\
x_{23}
\end{array}
\right],
\mathbf{b} =
\left[
\begin{array}{cccccc}
10 \\
15 \\
8 \\
12 \\
5
\end{array}
\right]
\] 

The determinant of $\mathbf{A}$, or $|\mathbf{A}|$ (or $det(\mathbf{A}))$, is 0. This is because the columns of $\mathbf{A}$ are linearly dependent on each other; e.g., column 1 = column 3 + column 4 - column 6. Hence, by considering the definition of total unimodularity, the transportation problem has equality constraints that are total unimodular \cite[Theorem 13.3, \S13.2, pp. 317]{Papadimitriou1998}.


%%%%%%%%%%%%%%%%%%%%%%%%%%%%%%%%%%%%%%%%%%%
\subsection{Integral Basic Solutions and Total Unimodularity}
\label{ssec:integralbasicsolutionsandtotalunimodularity}

Q1b): \\

From Equation \ref{eqn:transportationproblemexampleaxeqb}, the basic solutions are given as follows.

Let $x_{11} = x_{12} = 0$, and the set of equalities become:
\begin{eqnarray*}
\left[
\begin{array}{cccccc}
0 & 0 & 1 & 0 & 0 & 0 \\
0 & 0 & 0 & 1 & 1 & 1 \\
0 & 0 & 0 & 1 & 0 & 0 \\
0 & 0 & 0 & 0 & 1 & 0 \\
0 & 0 & 1 & 0 & 0 & 1
\end{array}
\right]
\left[
\begin{array}{cccccc}
x_{11} \\
x_{12} \\
x_{13} \\
x_{21} \\
x_{22} \\
x_{23}
\end{array}
\right]
=
\left[
\begin{array}{cccccc}
10 \\
15 \\
8 \\
12 \\
5
\end{array}
\right]
\end{eqnarray*}

Here, we have $x_{13} = 10$, $x_{22} = 12$, $x_{21} = 8$, $x_{23} = 5 - x_{13} = 5 - 10 = -5$. This is not a feasible solution.


Let $x_{11} = x_{13} = 0$, and the set of equalities become:
\begin{eqnarray*}
\left[
\begin{array}{cccccc}
0 & 0 & 1 & 0 & 0 & 0 \\
0 & 0 & 0 & 1 & 1 & 1 \\
0 & 0 & 0 & 1 & 0 & 0 \\
0 & 0 & 0 & 0 & 1 & 0 \\
0 & 0 & 1 & 0 & 0 & 1
\end{array}
\right]
\left[
\begin{array}{cccccc}
x_{11} \\
x_{12} \\
x_{13} \\
x_{21} \\
x_{22} \\
x_{23}
\end{array}
\right]
=
\left[
\begin{array}{cccccc}
10 \\
15 \\
8 \\
12 \\
5
\end{array}
\right]
\end{eqnarray*}

Here, we have $x_{12} = 15$, $x_{23} = 5$, $x_{21} = 8$, $x_{22} = 12 - x_{12} = 12 - 15 = -3$. This is not a feasible solution.


Let $x_{12} = x_{13} = 0$, and the set of equalities become:
\begin{eqnarray*}
\left[
\begin{array}{cccccc}
1 & 0 & 0 & 0 & 0 & 0 \\
0 & 0 & 0 & 1 & 1 & 1 \\
1 & 0 & 0 & 1 & 0 & 0 \\
0 & 0 & 0 & 0 & 1 & 0 \\
0 & 0 & 0 & 0 & 0 & 1
\end{array}
\right]
\left[
\begin{array}{cccccc}
x_{11} \\
x_{12} \\
x_{13} \\
x_{21} \\
x_{22} \\
x_{23}
\end{array}
\right]
=
\left[
\begin{array}{cccccc}
10 \\
15 \\
8 \\
12 \\
5
\end{array}
\right]\end{eqnarray*}

Here, we have $x_{11} = 10$, $x_{22} = 12$, $x_{23} = 5$, $x_{21} = 8 - x_{11} = 8 - 10 = -2$. This is not a feasible solution.



Let $x_{21} = x_{22} = 0$, and the set of equalities become:
\begin{eqnarray*}
\left[
\begin{array}{cccccc}
1 & 1 & 1 & 0 & 0 & 0 \\
0 & 0 & 0 & 0 & 0 & 1 \\
1 & 0 & 0 & 0 & 0 & 0 \\
0 & 1 & 0 & 0 & 0 & 0 \\
0 & 0 & 1 & 0 & 0 & 1
\end{array}
\right]
\left[
\begin{array}{cccccc}
x_{11} \\
x_{12} \\
x_{13} \\
x_{21} \\
x_{22} \\
x_{23}
\end{array}
\right]
=
\left[
\begin{array}{cccccc}
10 \\
15 \\
8 \\
12 \\
5
\end{array}
\right]\end{eqnarray*}

Here, we have $x_{23} = 15$, $x_{11} = 8$, $x_{12} = 12$, $x_{13} = 5 - x_{23} = 5 - 15 = -10$. This is not a feasible solution.




%%%%%%%%%%%%%%%%%%%%%%%%%%%
Let $x_{21} = x_{23} = 0$, and the set of equalities become:
\begin{eqnarray*}
\left[
\begin{array}{cccccc}
1 & 1 & 1 & 0 & 0 & 0 \\
0 & 0 & 0 & 0 & 1 & 0 \\
1 & 0 & 0 & 0 & 0 & 0 \\
0 & 1 & 0 & 0 & 1 & 0 \\
0 & 0 & 1 & 0 & 0 & 0
\end{array}
\right]
\left[
\begin{array}{cccccc}
x_{11} \\
x_{12} \\
x_{13} \\
x_{21} \\
x_{22} \\
x_{23}
\end{array}
\right]
=
\left[
\begin{array}{cccccc}
10 \\
15 \\
8 \\
12 \\
5
\end{array}
\right]\end{eqnarray*}

Here, we have $x_{22} = 15$, $x_{11} = 8$, $x_{13} = 5$, $x_{12} = 12 - x_{22} = 12 - 15 = -3$. This is not a feasible solution.




%%%%%%%%%%%%%%%%%%%%%%%%%%%
Let $x_{22} = x_{23} = 0$, and the set of equalities become:
\begin{eqnarray*}
\left[
\begin{array}{cccccc}
1 & 1 & 1 & 0 & 0 & 0 \\
0 & 0 & 0 & 1 & 0 & 0 \\
1 & 0 & 0 & 1 & 0 & 0 \\
0 & 1 & 0 & 0 & 0 & 0 \\
0 & 0 & 1 & 0 & 0 & 0
\end{array}
\right]
\left[
\begin{array}{cccccc}
x_{11} \\
x_{12} \\
x_{13} \\
x_{21} \\
x_{22} \\
x_{23}
\end{array}
\right]
=
\left[
\begin{array}{cccccc}
10 \\
15 \\
8 \\
12 \\
5
\end{array}
\right]\end{eqnarray*}

Here, we have $x_{21} = 15$, $x_{12} = 12$, $x_{13} = 5$, $x_{11} = 8 - x_{21} = 8 - 15 = -7$. This is not a feasible solution.


%%%%%%%%%%%%%%%%%%%%%%%%%%%
Let $x_{11} = x_{21} = 0$, and the set of equalities become:
\begin{eqnarray*}
\left[
\begin{array}{cccccc}
0 & 1 & 1 & 0 & 0 & 0 \\
0 & 0 & 0 & 0 & 1 & 1 \\
0 & 0 & 0 & 0 & 0 & 0 \\
0 & 1 & 0 & 0 & 1 & 0 \\
0 & 0 & 1 & 0 & 0 & 1
\end{array}
\right]
\left[
\begin{array}{cccccc}
x_{11} \\
x_{12} \\
x_{13} \\
x_{21} \\
x_{22} \\
x_{23}
\end{array}
\right]
=
\left[
\begin{array}{cccccc}
10 \\
15 \\
8 \\
12 \\
5
\end{array}
\right]\end{eqnarray*}

Here, we have $x_{22} + x_{23} = 15$, $x_{12} + x_{22} = 12$, $x_{13} + x_{23} = 5$, $x_{12} + x_{13} = 10$. Solving for the variables, we have integer values for the variables. Likewise, we get integer values when we set the following pair of variables to 0: $x_{11} = x_{22} = 0$, $x_{11} = x_{23} = 0$, $x_{12} = x_{21} = 0$, $x_{12} = x_{22} = 0$, $x_{12} = x_{23} = 0$, $x_{13} = x_{21} = 0$, $x_{13} = x_{22} = 0$, and $x_{13} = x_{23} = 0$.






















%	%%%%%%%%%%%%%%%%%%%%%%%%%%%
%	Let $x_{11} = x_{12} = 0$, and the set of equalities become:
%	\begin{eqnarray*}
%	\left[
%	\begin{array}{cccccc}
%	1 & 1 & 1 & 0 & 0 & 0 \\
%	0 & 0 & 0 & 1 & 1 & 1 \\
%	1 & 0 & 0 & 1 & 0 & 0 \\
%	0 & 1 & 0 & 0 & 1 & 0 \\
%	0 & 0 & 1 & 0 & 0 & 1
%	\end{array}
%	\right]
%	\left[
%	\begin{array}{cccccc}
%	x_{11} \\
%	x_{12} \\
%	x_{13} \\
%	x_{21} \\
%	x_{22} \\
%	x_{23}
%	\end{array}
%	\right]
%	=
%	\left[
%	\begin{array}{cccccc}
%	10 \\
%	15 \\
%	8 \\
%	12 \\
%	5
%	\end{array}
%	\right]\end{eqnarray*}
%	
%	Here, we have $x_{13} = 10$, $x_{22} = 12$, $x_{21} = 8$, $x_{23} = 5 - x_{13} = 5 - 10 = -5$. This is not a feasible solution.








Hence, based on the total unimodularity condition of $\mathbf{A}$, and given that all the elements of $\mathbf{A}$ and $\mathbf{b}$ are integers, all the basic solutions have integer components.


%%%%%%%%%%%%%%%%%%%%%%%%%%%%%%%%%%%%%%%%%%%
\section{MCNF-Klein's Algorithm}
\label{sec:kleinalgor}

Q2)

Q2.1) The computational complexity of creating the incremental weighted flow network (IWFN) or residual is $O(C)$, where $C$ represents the number of cycles in the network. Reference: \url{http://www.cs.princeton.edu/courses/archive/fall07/cos226/lectures/mincost.pdf} \\

Q2.2) The computational complexity of finding a negative cycle in an integrated weighted flow network (IWFN) is $O(E \cdot V)$, where $E$ represents the number of edges in the network and $V$ represents the number of nodes/vertices in the network. Reference: \url{http://www.cs.princeton.edu/courses/archive/fall07/cos226/lectures/mincost.pdf} \\

Q2.1) The computational complexity of Klein's cycle canceling algorithm is $O(E \cup C)$, where $C$ represents the number of cycles in the network and $E$ represents the number of edges in the network. Reference: \url{http://www.cs.princeton.edu/courses/archive/fall07/cos226/lectures/mincost.pdf} \\




%%%%%%%%%%%%%%%%%%%%%%%%%%%%%%%%%%%%%%%%%%%
\section{Problem Transformation}
\label{sec:problemtransformation}

Q3)

From Equation \ref{eqn:transportationproblem}, we can mathematically formulate the transportation problem as follows \cite[\S8.4, pp.439--447]{Miller2000}:
\begin{eqnarray}
	\begin{gathered}[b]
	{\rm minimize}\ T = \displaystyle\sum^{m}_{i = 1} \displaystyle\sum^{n}_{j = 1} t_{ij}x_{ij} \\
	\underline{x} \in S \\
	{\rm subject\ to:} \\
	\displaystyle\sum^{n}_{j = 1} x_{ij} = a_{i} \\
	\displaystyle\sum^{m}_{i = 1} x_{ij} = b_{j} \\
	x_{ij} \geq 0
	\end{gathered}
\end{eqnarray}

where $m$ is the number of places of origin for the supplies (i.e., goods produced), $n$ is the number of destinations for the supplies, $x_{ij}$ (the decision variable) is the quantity of goods send from a place of origin $i$ to its destination $j$, $t_{ij}$ is the cost of transporting each unit of supply/goods from origin $i$ to destination $j$, $a_{i}$ is the known supply at origin $i \in \{1, 2, \dots, m\}$, and $b_{j}$ is the known demand at destination $j \in \{1, 2, \dots, n\}$. \\

The minimum cost network flow problem (MCNF) can be mathematically formulated as follows in Equations \ref{eqn:mcnf}, \ref{eqn:flowbalanceeqn}, and \ref{eqn:limitationsonarccapacities} \cite[\S8.5, pp. 448--449; Summary, pp. 471--472]{Winston1995}:
\begin{eqnarray}
\label{eqn:mcnf}
	\begin{gathered}[b]
	{\rm minimize}\ C = \displaystyle\sum_{\rm all arcs} c_{ij}x_{ij} \\
	\underline{x} \in S \\
	{\rm subject\ to:}
	\end{gathered}
\end{eqnarray}

\begin{equation}
\label{eqn:flowbalanceeqn}
\displaystyle\sum_{j} x_{ij} - \displaystyle\sum_{k} x_{ki} = b_{i}
\end{equation}

\begin{equation}
\label{eqn:limitationsonarccapacities}
L_{ij} \leq x_{ij} \leq U_{ij}
\end{equation}

Here, $x_{ij}$ refers to the amount of units that flow from node $i$ to node $j$ through the $arc(i,j)$; $b_{i}$ is the net supply/flow (outflow - inflow) at node $i$; $c_{ij}$ is the unit flow cost of transportation from node $i$ to node $j$ through the $arc(i,j)$; $L_{ij}$ and $U_{ij}$ are respectively the lower and upper bounds on flow through the $arc(i,j)$; and $L_{ij} = \infty$ and $U_{ij} = \infty$ if the lower and upper bounds do not exist, respectively. Equation \ref{eqn:flowbalanceeqn} is the flow balance equation, and Equation \ref{eqn:limitationsonarccapacities} represent the constraints on the capacities of the arcs/edges. \\

The transportation problem can be formulated as a MCNF problem as follows \cite[\S8.5, pp. 448--450; Summary, pp. 471--472]{Winston1995}. Equation \ref{eqn:limitationsonarccapacities} does not apply for the transportation problem, since there are no arc capacity restrictions (i.e., $U_{ij} = \infty$). Equation \ref{eqn:flowbalanceeqn} is the flow balance equation, and it can be defined as demand (i.e., demand constraints) and supply (i.e., supply constraints). For the demand constraints, equations in the standard form is composed by summing up the $x_{ij}$'s for each destination node and equate it to its corresponding net supply/flow. Likewise, for the supply constraints, for each source node, determine the demand for goods/flow to create the equations representing these constraints. The objective function is the summation of the product of the cost per unit flow and number of unit flow on each arc. In cases where the total supply exceed the total demand, add dummy points to determine the net outflow at each supply point.

%%%%%%%%%%%%%%%%%%%%%%%%%%%%%%%%%%%%%%%%%%%
\section{Relax 4 for MCNF}
\label{sec:relax4}



%%%%%%%%%%%%%%%%%%%%%%%%%%%%%%%%%%%%%%%%%%%
\subsection{Q4.1}
\label{ssec:q4d1}


%%%%%%%%%%%%%%%%%%%%%%%%%%%%%%%%%%%%%%%%%%%
\subsection{Q4.2}
\label{ssec:q4d2}



%%%%%%%%%%%%%%%%%%%%%%%%%%%%%%%%%%%%%%%%%%%
\subsection{Q4.3}
\label{ssec:q4d3}


I prefer using Relax$^{\ast}$ IV to solve the minimum cost network flow problem, since it is easier to visualize and understand.


%%%%%%%%%%%%%%%%%%%%%%%%%%%%%%%%%%%%%%%%%%%%%
{\linespread{1}
\bibliographystyle{plain}
\bibliography{/data/research/antipastobibtex/references}
}
%%%%%%%%%%%%%%%%%%%%%%%%%%%%%%%%%%%%%%%%%%%%%
\end{document}