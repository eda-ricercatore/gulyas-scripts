\documentclass[letter,12pt]{article}
%%%%%%%%%%%%%%%%%%%%%%%%%%%%%%%%%%%%%%%%%%%%%%%%%
%	\usepackage{graphicx}
%	\usepackage{amsmath}
%	\usepackage{array}
%	\usepackage{amssymb}
%	\usepackage{setspace}
%	%\usepackage[margin=1.5cm,vmargin={0pt,1cm},nohead]{geometry}
%	\usepackage[margin=1in,vmargin={1in,1in}]{geometry}
%	% Package that has the symbol for ``:=''
%	\usepackage{txfonts}
%	% Create fancy headers and footers for this document
%	\usepackage{fancyhdr}
%	%\usepackage{cite}
%	% The ``cite'' package causes the hyperlinks for the in-text references/citations to fail. I believe it is because this package overrides the default package for referencing. Hence, only use the ``cite'' package with the IEEE format.
%	% Package for ``turnstile'' binary relations, where letters are defined above and below symbols
%	\usepackage{turnstile}
%	\usepackage{extarrows}
%	% Package that provides the cross symbol
%	\usepackage{ifsym}
%	\usepackage{marvosym}
%	% Commands for using the package for hyperlinks - 
%	\usepackage[pdftex,
%		pdftitle={Graphics and Color with LaTeX},
%		pdfauthor={Patrick W Daly},
%		pdfsubject={Importing images and use of color in LaTeX},
%		pdfkeywords={LaTeX, graphics, color},
%		pdfpagemode=UseOutlines,bookmarks, bookmarksopen,
%		pdfstartview=FitH, colorlinks, linkcolor=blue, citecolor=blue, urlcolor=red,
%	]{hyperref}
%	\hypersetup{colorlinks, linkcolor=blue}
%	% Concatenate references
%	\usepackage{cite}


%	% Package for tyepsetting algorithms and heuristics
%	\usepackage{listings}
%	\lstset{language=[GNU]C++}

%%%%%%%%%%%%%%%%%%%%%%%%%%%%%%%%%%%%%%%%%%%%%
%	Additional packages
\input{/data/others/grappanotes/others/preamble}
%	AMS theorem package
\usepackage{amsthm}
%\usepackage{lscape}
%\usepackage{rotating}
\usepackage{pdflscape}


% definition of new \LaTeX command for the citation: \cite{Cimatti08} and \cite{Barrett09}
% This allows mathematical/logic symbols to be typeset with the font ``Zapf Chancery'' in ``\LaTeX\ math mode''. To typeset symbols in such font, try: \mathpzc{ABCdef123}
\DeclareMathAlphabet{\mathpzc}{OT1}{pzc}{m}{it}

%%%%%%%%%%%%%%%%%%%%%%%%%%%%%%%%%%%%%%%%%%%%%
% Start of document
\begin{document}
\title{Discrete Optimization via Max-SMT}
\date{\today}
\author{Jack Jun-Tao Yang 
	\thanks{Email correspondence to: \href{mailto:havachoice@gmail.com}{havachoice@gmail.com}}
	and Zhiyang Ong
	\thanks{Email correspondence to: \href{mailto:ongz@acm.org}{ongz@acm.org}}
}
\maketitle



\begin{abstract}
%	Problems in mathematical programming can be formulated as a conjunction of its objective function(s) and constraint(s), and solved with decision procedures for problems in maximum satisfiability modulo theories (Max-SMT) {\LARGE Insert Reference!!!}. This is because software for satisfiability modulo theories (SMT) can reason on first-order logic formulae based the theories of nonlinear real and integer arithmetic, linear real and integer arithmetic, quantified first-order logic {\LARGE Insert Reference!!!}. For the class project of my ``Discrete Optimization'' class at National Taiwan University in Fall 2013, I propose developing a flow for discrete optimization based on Max-SMT solvers. The milestones for the class project are described as follows: encoding linear programming problems in standard form as SMT formulae, using the SMT-LIB 2 format {\LARGE Insert Reference!!!}, and solving the linear programming problems with existing open-source Max-SAT solvers (solvers for the maximum satisfiability problem, Max-SAT) {\LARGE Insert Reference!!!}; extend {\sc MiniSat} to solve problems in Max-SAT; evaluate current open-source SMT solvers (CVC4 and OpenSMT) {\LARGE Insert Reference!!!} for extension to Max-SMT; and, lastly, implement the extension of an SMT solver to Max-SMT. The last milestone would enable us to extend discrete optimization to discrete nonlinear programming (i.e., nonlinear integer programming) problems.
Problems in mathematical programming can be formulated as a conjunction of its objective function(s) and constraint(s), and solved with decision procedures for problems in maximum satisfiability modulo theories (Max-SMT) {\LARGE Insert Reference!!!}. This is because software for satisfiability modulo theories (SMT) can reason on first-order logic formulae based the theories of nonlinear real and integer arithmetic, linear real and integer arithmetic, quantified first-order logic {\LARGE Insert Reference!!!}. For the class project of my ``Discrete Optimization'' class at National Taiwan University in Fall 2013, I propose developing a flow for discrete optimization based on Max-SMT solvers. The milestones for the class project are described as follows: encoding linear programming (LP) problems in standard form as SMT formulae, using the SMT-LIB 2 format {\LARGE Insert Reference!!!}, and solving the linear programming problems with existing open-source Max-SAT solvers (solvers for the maximum satisfiability problem, Max-SAT) and SMT solvers {\LARGE Insert Reference!!!}; use the SMT solver to solve integer linear programming (ILP) problems; compare different approaches to solve ILP problems; and create a meta-algorithm for algorithmic portfolio optimization.
\end{abstract}

%	%%%%%%%%%%%%%%%%%%%%%%%%%%%%%%%%%%%%%%%%%%%
%	\section*{Declaration}
%	\label{sec:declaration}
%	
%	I did this assignment on my own without any collaborators. %The mathematical programming package that I have used is \cite{Makhorin2012}. 







%%%%%%%%%%%%%%%%%%%%%%%%%%%%%%%%%%%%%%%%%%%
\section{Objectives of Study}
\label{sec:objectivesofstudy}

The objectives for our term project are: \vspace{-0.3cm}
\begin{enumerate} \itemsep -4pt
\item To help 
\end{enumerate}



%%%%%%%%%%%%%%%%%%%%%%%%%%%%%%%%%%%%%%%%%%%
\section{Summary of Literature Survey}
\label{sec:summaryofliteraturesurvey}



%%%%%%%%%%%%%%%%%%%%%%%%%%%%%%%%%%%%%%%%%%%
\section{Research Proposal}
\label{sec:summaryofliteraturesurvey}

The progress of modern SAT solvers can be shown in \cite[Fig. 1.2, pp. 5, Chapter 1]{Samulowitz2008} \cite[slide 7]{Sabharwal2011} \cite[slide 7]{Sabharwal2007}.

\cite[slide 9]{Ganesh2013}






The reference used was: \cite[\S2.5, pages 25--27]{Luenberger2008}. \\

The set of given constraints are:
\begin{eqnarray*}
x_{1} + \frac{8}{3}x_{2} \leq 4 \\
x_{1} + x_{2} \leq 2 \\
2x_{1} \leq 3 \\
x_{1} \geq 0, x_{2} \geq 0
\end{eqnarray*}



To solve this via a linear programming approach, use slack variables to convert the set of inequalities in $E^{2}$ to an equivalent set of equalities in $E^{5}$, where the five extreme points of the feasible set can be see in Figure \ref{fig:q1feasibleset} \cite[\S2.1, pages 12]{Luenberger2008}. \\

Thus, the resultant set of equalities, represented as $\mathbf{Ax = b}$ \cite[\S2.1, page 19]{Luenberger2008}, in the standard form \cite[\S2.1, pages 11]{Luenberger2008} of linear programming is given as follows:
\begin{eqnarray*}
x_{1} + \frac{8}{3}x_{2} + x_{3} = 4 \\
x_{1} + x_{2} + x_{4} = 2 \\
2x_{1} + x_{5} = 3 \\
x_{1} \geq 0, x_{2} \geq 0, x_{3} \geq 0, x_{4} \geq 0, x_{5} \geq 0
\end{eqnarray*}

The basic solutions for this system of linear equations can be found by ``setting any two of variables to zero and solving for the remaining three'' \cite[\S2.5, page 26]{Luenberger2008}. \\

Possible Basic solution \#1:
Let $x_{1} = x_{2} = 0$, and the set of equalities become:
\begin{eqnarray*}
0 + \frac{8}{3} \times 0 + x_{3} = 4 \\
0 + 0 + x_{4} = 2 \\
2 \times 0 + x_{5} = 3 \\
\end{eqnarray*}
Therefore, for possible basic solution \#1, $x_{1} = x_{2} = 0, x_{3} = 4, x_{4} = 2, x_{5} = 3$:
\begin{equation}
\mathbf{x} = \left[
	\begin{array}{c}
	0 \\ 0 \\ 4 \\ 2 \\ 3
	\end{array}
	\right]
\end{equation}


%%%%%%%%%%%%%%%%%%%%%%%%%%%%%%%%%%%%%%%%%%%
\section{Transforming Optimization Problem with Piecewise Linear Functions into Linear Programming Problem}
\label{sec:optimizationproblemtransformation}

The given mathematical optimization problem for a given class of piecewise linear functions is:
\begin{eqnarray*}
f({\mathbf{x}}) = {\rm maximum} (\mathbf{c}^{T}_{1}\mathbf{x} + d_{1}, \mathbf{c}^{T}_{2}\mathbf{x} + d_{2}, \dots, \mathbf{c}^{T}_{p}\mathbf{x} + d_{p})
\end{eqnarray*}

For this I can convert it into a linear model by separating the $x_{i}$ and $d_{i}$ components, $\forall i \in {1, 2, ..., n}$
programming \cite[\S3.3, pages 83--86]{Haftka1992}\cite[\S1, pages 5-8]{Malucelli2005}. Add all the $x_{i}$ components and place them into the new objective function with the initial displacement $d_{1}$. For each additive/displacement term $d_{i}$ that causes the model to be piecewise linear, multiply it by a slack variables and constraint them within a range $d_{i}y_{i1} \leq z_{i} \leq d_{i}y_{i2}$ That is, re-formulate the problem as:
\begin{eqnarray*}
f({\mathbf{x}}) = {\rm maximum} (d_{1}y_{1} + \mathbf{c}^{T}_{1}\mathbf{x} + \mathbf{c}^{T}_{2}\mathbf{x} + \dots + \mathbf{c}^{T}_{p}\mathbf{x}) \\
{\rm subject\ to:} \\
%d_{2}y_{1a} \leq z_{1} \leq d_{1}y_{1b} \\
d_{2}y_{2a} \leq z_{2} \leq d_{2}y_{2b} \\
d_{3}y_{3a} \leq z_{3} \leq d_{3}y_{3b} \\
\dots \\
d_{p}y_{pa} \leq z_{p} \leq d_{p}y_{pb} \\
\end{eqnarray*}


Subsequently, I can convert the maximization problem in linear programming into a minimization problem if linear programming in the standard form \cite[\S3.4, page 48, Example 1]{Luenberger2008}. Negate ``the objective function'' to convert it to a minimization problem. For each constraint, add a unique slack variable to turn them into equalities in the standard form of linear programming.

%%%%%%%%%%%%%%%%%%%%%%%%%%%%%%%%%%%%%%%%%%%
\section{Simplex Method}
\label{ssec:simplexmethod}



%%%%%%%%%%%%%%%%%%%%%%%%%%%%%%%%%%%%%%%%%%%
\subsection{Simplex Method, Implemented Using Linear Algebra}
\label{ssec:simplexmethodlinearalgebra}

\begin{eqnarray*}
{\rm maximize}\ (-x_{1} + x_{2}) \\
{\rm subject\ to} \\
x_{1} - x_{2} \leq 2 \\
x_{1} + x_{2} \leq 6 \\
x_{1} \geq 0, x_{1} \geq 0 \\
\end{eqnarray*}

Transform this to normal form by introducing slack variables $x_{3}$ and $x_{4}$, the objective/maximization function and constraints become: \\
\begin{eqnarray*}
z - x_{1} + x_{2} = 0 \\
x_{1} - x_{2} + x_{3} = 2 \\
x_{1} + x_{2} + x_{4 } = 6 \\
{\rm where}\ x_{1} \geq 0, x_{1} \geq 0 \\
\end{eqnarray*}


In the augmented matrix form (of $\mathbf{Ax = b}$), it becomes:
\[  \mathbf{T_{0}} =  
\begin{array}{c}
	\begin{array}{cccccc}
	z & x_{1} & x_{2} & x_{3} & x_{4} & b \\
	\end{array}
	\\
	\left[ \begin{array}{cccccc}
	1 & -1 & 1 & 0 & 0 & 0 \\
	0 & 1 & -1 & 1 & 0 & 2  \\
	0 & 1 & 1 & 0 & 1 & 6
	\end{array} \right]
\end{array}
\]

Select as the column of the pivot the first column with a negative entry in Row 1. This is column 3, because of the $-1$.\\

Selection of the row of the pivot. Divide the right sides (2 and 6) by the corresponding entries of the selected column ($2 \div -1 = -2$ and $6/1 = 6$). The smaller quotient of the rightmost column is the second row, with the quotient of $-2$.\\

\begin{equation*}
\mathbf{T_{1}} =  
\begin{array}{c}
	\begin{array}{cccccc}
	z & x_{1} & x_{2} & x_{3} & x_{4} & b \\
	\end{array}
	\\
	\left[ \begin{array}{cccccc}
	1 & -1 & 1 & 0 & 0 & 0 \\
	0 & 1 & -1 & 1 & 0 & -2  \\
	0 & 1 & 1 & 0 & 1 & 6
	\end{array} \right]
\end{array}
\end{equation*}

Elimination by row operations. Replace Row 1 with the addition of Row 1 and Row 3.

\begin{equation*}
\mathbf{T_{2}} =  
\begin{array}{c}
	\begin{array}{cccccc}
	z & x_{1} & x_{2} & x_{3} & x_{4} & b \\
	\end{array}
	\\
	\left[ \begin{array}{cccccc}
	1 & 0 & 2 & 0 & 1 & 6\ {\rm (Row\ 1 + Row\ 3)} \\
	0 & 1 & -1 & 1 & 0 & -1  \\
	0 & 1 & 1 & 0 & 1 & 6
	\end{array} \right]
\end{array}
\end{equation*}

Since there are no more negative entries in Row 1 of $T_{2}$, the Simplex method terminates with an optimum value of 6, which is the last column entry in Row 1. \\

That is, the optimum solution to the linear programming problem is 6.



The last milestone would enable us to extend discrete optimization to discrete nonlinear programming (i.e., nonlinear integer programming) problems.





%%%%%%%%%%%%%%%%%%%%%%%%%%%%%%%%%%%%%%%%%%%
\section{Task Division Among Team Members}
\label{sec:taskdiv}

The project members are Zhiyang Ong and Jack Jun-Tao Yang. The following sections indicate: the list of milestones and tasks to be completed for the project (see \S\ref{ssec:tasklist}); and the project schedule (Gantt chart (insert reference)), resource utilization chart, and PERT chart (insert reference) (\S\ref{ssec:projscheduleandresourcenpertcharts}).






%%%%%%%%%%%%%%%%%%%%%%%%%%%%%%%%%%%%%%%%%%%
\subsection{Task List}
\label{ssec:tasklist}

Table \ref{tab:tasklisttable} shows a list of tasks and milestones that must be completed for the class project.


\begin{table}%[htdp]
\caption{List of tasks (and milestones) that must be completed for the class project.}
\begin{center}
	\begin{tabular}[b]{|c|c|c|c|c|c|c|}\hline
	\label{tab:tasklisttable}
		Name & Start & End & Milestone & Percentage & Resources & Notes \\
		 & Date & Date &  & Completed &  &  \\
		\hline
		Project Proposal & September 1 & Nov 22 & \checkmark & 5\% & Zhiyang & \\
		 &  &  &  &  & and Jack &  \\
		 \hline
		Optimization & Nov 1 & Nov 22 & \checkmark & 1\% & Zhiyang & Delayed \\
		via Max-SAT &  &  &  &  & and Jack &  \\
		and SMT solving &  &  &  &  &  &  \\
		\hline
		Numerical & Oct 29 & Nov 20 &  & 5\% & Zhiyang & In progress \\
		Approximation &  &  &  &  & and Jack &  \\
		(in Python) &  &  &  &  &  &  \\
		\hline
		Mathematical & Nov 18 & Dec 5 & \checkmark & 0\% & Zhiyang &  \\
		Optimization &  &  &  &  & and Jack &  \\
		(in Python) &  &  &  &  &  &  \\
		\hline
		Portfolio & Nov 14 & Dec 18 & \checkmark & 1\% & Zhiyang & Ongoing \\
		Optimization &  &  &  &  & and Jack &  \\
		\hline
		Project Report & Dec 18 & Jan 8 & \checkmark & 0\% & Zhiyang & \\
		 &  &  &  &  & and Jack &  \\
	\hline
	\end{tabular}
\end{center}
\end{table}





%%%%%%%%%%%%%%%%%%%%%%%%%%%%%%%%%%%%%%%%%%%
\subsection{Project Schedule, Resource Chart, and PERT Chart}
\label{ssec:projscheduleandresourcenpertcharts}

%	3 figures concatenated vertically
%	\begin{landscape}
%	\centering
%		\begin{figure}[ht]
%		\begin{minipage}[b]{9.5in}
%		\setlength{\unitlength}{0.2in}
%		\includegraphics[width=9.5in]{./gantt_chart/gantt_chart}
%		\caption{A Gantt chart showing the project schedule, and duration for each task.}
%		\label{fig:ganttchart}
%		\end{minipage}
%	
%		\vspace{1in}
%		
%		\begin{minipage}[b]{9.5in}
%		\setlength{\unitlength}{0.2in}
%		\includegraphics[width=9.5in]{./gantt_chart/resources_chart}
%		\caption{A resource chart indicating how team members are being allocated tasks during the project.}
%		\label{fig:resourceschart}
%		\end{minipage}
%		
%		\begin{minipage}[b]{9.5in}
%		\setlength{\unitlength}{0.2in}
%		\includegraphics[width=7.5in]{./gantt_chart/pert_chart}
%		\caption{PERT chart indicate the tasks/milestones for the project.}
%		\label{fig:pertchart}
%		\end{minipage}
%		
%		\end{figure}
%	\end{landscape}

\begin{landscape}
\centering
	\begin{figure}[ht]
	\begin{minipage}[b]{9.5in}
	\setlength{\unitlength}{0.2in}
%	\centering
	\includegraphics[width=9.5in]{./gantt_chart/gantt_chart}
	\caption{A Gantt chart showing the project schedule, and duration for each task.}
	\label{fig:ganttchart}
	\end{minipage}

	\vspace{2in}
	
	\begin{minipage}[b]{9.5in}
	\setlength{\unitlength}{0.2in}
%	\centering
	\includegraphics[width=9.5in]{./gantt_chart/resources_chart}
	\caption{A resource chart indicating how team members are being allocated tasks during the project.}
	\label{fig:resourceschart}
	\end{minipage}
	
	\end{figure}
\end{landscape}


\begin{figure}[h]
\centering 
\includegraphics[width=7.5in]{./gantt_chart/pert_chart}
\caption{PERT chart indicate the tasks/milestones for the project.}
\label{fig:pertchart}
\end{figure}







%%%%%%%%%%%%%%%%%%%%%%%%%%%%%%%%%%%%%%%%%%%%%
{\linespread{1}
\bibliographystyle{plain}
\bibliography{/data/research/antipastobibtex/references}
}
%%%%%%%%%%%%%%%%%%%%%%%%%%%%%%%%%%%%%%%%%%%%%
\end{document}