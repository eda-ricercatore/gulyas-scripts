\documentclass[letter,12pt]{article}
%%%%%%%%%%%%%%%%%%%%%%%%%%%%%%%%%%%%%%%%%%%%%%%%%
%	\usepackage{graphicx}
%	\usepackage{amsmath}
%	\usepackage{array}
%	\usepackage{amssymb}
%	\usepackage{setspace}
%	%\usepackage[margin=1.5cm,vmargin={0pt,1cm},nohead]{geometry}
%	\usepackage[margin=1in,vmargin={1in,1in}]{geometry}
%	% Package that has the symbol for ``:=''
%	\usepackage{txfonts}
%	% Create fancy headers and footers for this document
%	\usepackage{fancyhdr}
%	%\usepackage{cite}
%	% The ``cite'' package causes the hyperlinks for the in-text references/citations to fail. I believe it is because this package overrides the default package for referencing. Hence, only use the ``cite'' package with the IEEE format.
%	% Package for ``turnstile'' binary relations, where letters are defined above and below symbols
%	\usepackage{turnstile}
%	\usepackage{extarrows}
%	% Package that provides the cross symbol
%	\usepackage{ifsym}
%	\usepackage{marvosym}
%	% Commands for using the package for hyperlinks - 
%	\usepackage[pdftex,
%		pdftitle={Graphics and Color with LaTeX},
%		pdfauthor={Patrick W Daly},
%		pdfsubject={Importing images and use of color in LaTeX},
%		pdfkeywords={LaTeX, graphics, color},
%		pdfpagemode=UseOutlines,bookmarks, bookmarksopen,
%		pdfstartview=FitH, colorlinks, linkcolor=blue, citecolor=blue, urlcolor=red,
%	]{hyperref}
%	\hypersetup{colorlinks, linkcolor=blue}
%	% Concatenate references
%	\usepackage{cite}


%	% Package for tyepsetting algorithms and heuristics
%	\usepackage{listings}
%	\lstset{language=[GNU]C++}

%%%%%%%%%%%%%%%%%%%%%%%%%%%%%%%%%%%%%%%%%%%%%
%	Additional packages
\input{/data/others/grappanotes/others/preamble}
%	AMS theorem package
\usepackage{amsthm}




% definition of new \LaTeX command for the citation: \cite{Cimatti08} and \cite{Barrett09}
% This allows mathematical/logic symbols to be typeset with the font ``Zapf Chancery'' in ``\LaTeX\ math mode''. To typeset symbols in such font, try: \mathpzc{ABCdef123}
\DeclareMathAlphabet{\mathpzc}{OT1}{pzc}{m}{it}

%%%%%%%%%%%%%%%%%%%%%%%%%%%%%%%%%%%%%%%%%%%%%
% Start of document
\begin{document}
\title{Scribed Notes for Week 2: September 17, 2013}
\date{\today}
\author{Zhiyang Ong
	\thanks{Email correspondence to: \href{mailto:ongz@acm.org}{ongz@acm.org}}
}
\maketitle






%%%%%%%%%%%%%%%%%%%%%%%%%%%%%%%%%%%%%%%%%%%
\section{Declaration}
\label{sec:declaration}

I did this assignment on my own without any collaborators. The mathematical programming package that I have used is \cite{Makhorin2012}. 







%%%%%%%%%%%%%%%%%%%%%%%%%%%%%%%%%%%%%%%%%%%
\section{Finding the Perfect Diet}
\label{sec:findingperfectdiet}

Q1.1) and Q1.2): See printouts. \\

Q1.3) The minimal cost from CPLEX is 92.5, and the minimal cost from GLPK is about 80.69. Since GLPK yields a lower cost that CPLEX, it gives a better solution to the linear programming problem.


%%%%%%%%%%%%%%%%%%%%%%%%%%%%%%%%%%%%%%%%%%%
\section{Finding the Optimal Volume Mix of Two Types of Liquid}
\label{sec:optimalliquidmix}

Q2.1) Mathematical formulation for finding the optimal volume mix of two types of liquid. \\

Let $x_{1}$ and $x_{2}$ represent the volume of liquid types 1 and 2 in cm$^{3}$. \\

Objective function:

maximize $F(x) = \$2/{\rm cm}^{3} \times x_{1} + \$3/{\rm cm}^{3} \times x_{2} $ \\

Decision variables:

$\underline{x}_{\varepsilon} s$ \\

Subject to these constraints:

$x_{1} + x_{2} \leq 2000 $ (since 2 $l = 2000$ cm$^{3}$)

$1 g/cm^{3} \times x_{1} + 2 g/cm^{3} \times x_{2} \leq 3000 $ (since 3 kg $ = 3000$ g)

$x_{1} \geq 0$

$x_{2} \geq 0$
\ \\
\ \\
Here, I define density $\rho$ = $\frac{{\rm mass}\ m}{{\rm volume}\ v}$. \\

$\therefore m = \frac{\rho}{v}$


Q2.2) See printout.


%%%%%%%%%%%%%%%%%%%%%%%%%%%%%%%%%%%%%%%%%%%
\section{Nonlinear Programming}
\label{sec:nonlinearprogramming}

Q3) Attachment




%%%%%%%%%%%%%%%%%%%%%%%%%%%%%%%%%%%%%%%%%%%
\section{Traveling Salesperson Problem}
\label{sec:tspquestion}

%For the asymmetric traveling salesperson problem (TSP), the distance, $d$, between any pair of cities $i$ and $j$ is not equivalent. That is, $d(i,j) \neq d(j,i)$).

%The number of tours for asymmetric TSP is: $(n-1)!$ \cite[\S6.1.1, pp. 224]{Punnen2007}. Therefore, the cardinality for $N_{2}(t)$ is: $\frac{1}{2}(n-2)!$ % $\frac{1}{2}(n-2)!$

Q4.1) The cardinality for $N_{2}(t)$ is: n. \\
\ \\
Q4.2) $N_{2}(t) = \{ t: t \in T \& t can be obtained by removing 2 edges from the tour and then replacing them with 2 edges \}$















%%%%%%%%%%%%%%%%%%%%%%%%%%%%%%%%%%%%%%%%%%%%%
{\linespread{1}
\bibliographystyle{plain}
\bibliography{/data/research/antipastobibtex/references}
}
%%%%%%%%%%%%%%%%%%%%%%%%%%%%%%%%%%%%%%%%%%%%%
\end{document}