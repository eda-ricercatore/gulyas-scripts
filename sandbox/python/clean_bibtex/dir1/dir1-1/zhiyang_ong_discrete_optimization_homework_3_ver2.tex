\documentclass[letter,12pt]{article}
%%%%%%%%%%%%%%%%%%%%%%%%%%%%%%%%%%%%%%%%%%%%%%%%%
%	\usepackage{graphicx}
%	\usepackage{amsmath}
%	\usepackage{array}
%	\usepackage{amssymb}
%	\usepackage{setspace}
%	%\usepackage[margin=1.5cm,vmargin={0pt,1cm},nohead]{geometry}
%	\usepackage[margin=1in,vmargin={1in,1in}]{geometry}
%	% Package that has the symbol for ``:=''
%	\usepackage{txfonts}
%	% Create fancy headers and footers for this document
%	\usepackage{fancyhdr}
%	%\usepackage{cite}
%	% The ``cite'' package causes the hyperlinks for the in-text references/citations to fail. I believe it is because this package overrides the default package for referencing. Hence, only use the ``cite'' package with the IEEE format.
%	% Package for ``turnstile'' binary relations, where letters are defined above and below symbols
%	\usepackage{turnstile}
%	\usepackage{extarrows}
%	% Package that provides the cross symbol
%	\usepackage{ifsym}
%	\usepackage{marvosym}
%	% Commands for using the package for hyperlinks - 
%	\usepackage[pdftex,
%		pdftitle={Graphics and Color with LaTeX},
%		pdfauthor={Patrick W Daly},
%		pdfsubject={Importing images and use of color in LaTeX},
%		pdfkeywords={LaTeX, graphics, color},
%		pdfpagemode=UseOutlines,bookmarks, bookmarksopen,
%		pdfstartview=FitH, colorlinks, linkcolor=blue, citecolor=blue, urlcolor=red,
%	]{hyperref}
%	\hypersetup{colorlinks, linkcolor=blue}
%	% Concatenate references
%	\usepackage{cite}


%	% Package for tyepsetting algorithms and heuristics
%	\usepackage{listings}
%	\lstset{language=[GNU]C++}

%%%%%%%%%%%%%%%%%%%%%%%%%%%%%%%%%%%%%%%%%%%%%
%	Additional packages
\input{/data/others/grappanotes/others/preamble}
%	AMS theorem package
\usepackage{amsthm}




% definition of new \LaTeX command for the citation: \cite{Cimatti08} and \cite{Barrett09}
% This allows mathematical/logic symbols to be typeset with the font ``Zapf Chancery'' in ``\LaTeX\ math mode''. To typeset symbols in such font, try: \mathpzc{ABCdef123}
\DeclareMathAlphabet{\mathpzc}{OT1}{pzc}{m}{it}

%%%%%%%%%%%%%%%%%%%%%%%%%%%%%%%%%%%%%%%%%%%%%
% Start of document
\begin{document}
\title{Discrete Optimization Homework \#3}
\date{\today}
\author{Zhiyang Ong
	\thanks{Email correspondence to: \href{mailto:ongz@acm.org}{ongz@acm.org}}
}
\maketitle






%%%%%%%%%%%%%%%%%%%%%%%%%%%%%%%%%%%%%%%%%%%
\section*{Declaration}
\label{sec:declaration}

I did this assignment on my own without any collaborators. %The mathematical programming package that I have used is \cite{Makhorin2012}. 







%%%%%%%%%%%%%%%%%%%%%%%%%%%%%%%%%%%%%%%%%%%
\section{Finding Basic Solutions}
\label{sec:findingbasicsolutions}

The reference used was: \cite[\S2.5, pages 25--27]{Luenberger2008}. \\

The set of given constraints are:
\begin{eqnarray*}
x_{1} + \frac{8}{3}x_{2} \leq 4 \\
x_{1} + x_{2} \leq 2 \\
2x_{1} \leq 3 \\
x_{1} \geq 0, x_{2} \geq 0
\end{eqnarray*}

\begin{figure}
\centering 
\includegraphics[width=6in]{q1_fig}
\caption{Feasible set for the given constraints \cite[\S2.5, page 27]{Luenberger2008}}
\label{fig:q1feasibleset}
\end{figure}

To solve this via a linear programming approach, use slack variables to convert the set of inequalities in $E^{2}$ to an equivalent set of equalities in $E^{5}$, where the five extreme points of the feasible set can be see in Figure \ref{fig:q1feasibleset} \cite[\S2.1, pages 12]{Luenberger2008}. \\

Thus, the resultant set of equalities, represented as $\mathbf{Ax = b}$ \cite[\S2.1, page 19]{Luenberger2008}, in the standard form \cite[\S2.1, pages 11]{Luenberger2008} of linear programming is given as follows:
\begin{eqnarray*}
x_{1} + \frac{8}{3}x_{2} + x_{3} = 4 \\
x_{1} + x_{2} + x_{4} = 2 \\
2x_{1} + x_{5} = 3 \\
x_{1} \geq 0, x_{2} \geq 0, x_{3} \geq 0, x_{4} \geq 0, x_{5} \geq 0
\end{eqnarray*}

The basic solutions for this system of linear equations can be found by ``setting any two of variables to zero and solving for the remaining three'' \cite[\S2.5, page 26]{Luenberger2008}. \\

Possible Basic solution \#1:
Let $x_{1} = x_{2} = 0$, and the set of equalities become:
\begin{eqnarray*}
0 + \frac{8}{3} \times 0 + x_{3} = 4 \\
0 + 0 + x_{4} = 2 \\
2 \times 0 + x_{5} = 3 \\
\end{eqnarray*}
Therefore, for possible basic solution \#1, $x_{1} = x_{2} = 0, x_{3} = 4, x_{4} = 2, x_{5} = 3$:
\begin{equation}
\mathbf{x} = \left[
	\begin{array}{c}
	0 \\ 0 \\ 4 \\ 2 \\ 3
	\end{array}
	\right]
\end{equation}

\ \\
\ \\

Possible Basic solution \#2:
Let $x_{1} = x_{3} = 0$, and the set of equalities become:
\begin{eqnarray*}
0 + \frac{8}{3} \times x_{2} + 0 = 4 \\
0 + x_{2} + x_{4} = 2 \\
2 \times 0 + x_{5} = 3 \\
\end{eqnarray*}
Therefore, for possible basic solution \#2, $x_{1} = x_{3} = 0, \frac{8}{3} x_{2} = 4 \Rightarrow x_{2} = 4 \times \frac{3}{8} = 6, x_{4} = 2 - x_{2} = 2 - 6 = -4, x_{5} = 3$:
\begin{equation}
\mathbf{x} = \left[
	\begin{array}{c}
	0 \\ 6 \\ 0 \\ -4 \\ 3
	\end{array}
	\right]
\end{equation}
(reject) \\
\ \\
\ \\

Possible Basic solution \#3:
Let $x_{1} = x_{4} = 0$, and the set of equalities become:
\begin{eqnarray*}
0 + \frac{8}{3}x_{2} + x_{3} = 4 \\
0 + x_{2} + 0 = 2 \\
0 + x_{5} = 3 \\
\end{eqnarray*}
Therefore, for possible basic solution \#3, $x_{1} = x_{4} = 0, x_{2} = 2, x_{3} = 4 - \frac{8}{3} x_{2} = 4 - \frac{3}{8} \times 2 = 3\frac{1}{4}, x_{5} = 3$:
\begin{equation}
\mathbf{x} = \left[
	\begin{array}{c}
	0 \\ 2 \\ 3.25 \\ 0 \\ 3
	\end{array}
	\right]
\end{equation}
(reject) \\
\ \\
\ \\

Possible Basic solution \#4:
Let $x_{2} = x_{3} = 0$, and the set of equalities become:
\begin{eqnarray*}
x_{1} + \frac{8}{3} \times 0 + 0 = 4 \\
x_{1} + 0 + x_{4} = 2 \\
2x_{1} + x_{5} = 3 \\
\end{eqnarray*}
Therefore, for possible basic solution \#4, $x_{2} = x_{3} = 0, x_{1} = 4, x_{4} = 2 - x_{1} = 2 - 4 = -2, x_{5}  = 3 - 2 \times x_{1} = 3 - 2 \times 4 = -5$:
\begin{equation}
\mathbf{x} = \left[
	\begin{array}{c}
	4 \\ 0 \\ 0 \\ -2 \\ -5
	\end{array}
	\right]
\end{equation}
(reject) \\
\ \\
\ \\

Possible Basic solution \#5:
Let $x_{2} = x_{4} = 0$, and the set of equalities become:
\begin{eqnarray*}
x_{1} + 0 + x_{3} = 4 \\
x_{1} + 0 + 0 = 2 \\
2x_{1} + x_{5} = 3 \\
\end{eqnarray*}
Therefore, for possible basic solution \#5, $x_{2} = x_{4} = 0, x_{1} = 2, x_{3} = 4 - x_{1} = 4 - 2 = 2, x_{5} = 3 - 2 \times x_{1} = 3 - 2 \times 2 = -1$:
\begin{equation}
\mathbf{x} = \left[
	\begin{array}{c}
	2 \\ 0 \\ 2 \\ 0 \\ -1
	\end{array}
	\right]
\end{equation}
(reject) \\
\ \\
\ \\

Possible Basic solution \#6:
Let $x_{2} = x_{5} = 0$, and the set of equalities become:
\begin{eqnarray*}
x_{1} + 0 + x_{3} = 4 \\
x_{1} + 0 + x_{4} = 2 \\
2x_{1} + 0 = 3 \\
\end{eqnarray*}
Therefore, for possible basic solution \#6, $x_{2} = x_{5} = 0, x_{1} = \frac{3}{2} = 1.5, x_{3} = 4 - x_{1} = 4 - 1.5 = 2.5, x_{4} = 2 - x_{1} = 2 - 1.5 = 0.5$:
\begin{equation}
\mathbf{x} = \left[
	\begin{array}{c}
	1.5 \\ 0 \\ 2.5 \\ 0.5 \\ 0
	\end{array}
	\right]
\end{equation}
(reject) \\
\ \\
\ \\

Possible Basic solution \#7:
Let $x_{3} = x_{4} = 0$, and the set of equalities become:
\begin{eqnarray*}
x_{1} + \frac{8}{3}x_{2} + 0 = 4 \\
x_{1} + x_{2} + 0 = 2 \\
2x_{1} + x_{5} = 3 \\
\end{eqnarray*}
Therefore, for possible basic solution \#7, $x_{3} = x_{4} = 0, (\frac{8}{3} - 1) \times x_{2} = 4 - 2 \Rightarrow x_{2} = 2 \times \frac{3}{5} = 1.2, x_{1} = 2 - x_{1} = 2 - 1.2 = 0.8, x_{5} = 3 - 2 \times x_{1} = 0.6$:
\begin{equation}
\mathbf{x} = \left[
	\begin{array}{c}
	0.8 \\ 1.2 \\ 0 \\ 0 \\ 0.6
	\end{array}
	\right]
\end{equation}
(reject) \\
\ \\
\ \\

Possible Basic solution \#8:
Let $x_{3} = x_{5} = 0$, and the set of equalities become:
\begin{eqnarray*}
x_{1} + \frac{8}{3}x_{2} + 0 = 4 \\
x_{1} + x_{2} + x_{4} = 2 \\
2x_{1} + 0 = 3 \\
\end{eqnarray*}
Therefore, for possible basic solution \#8, $x_{3} = x_{5} = 0, x_{1} = \frac{3}{2} = 1.5, x_{2} = \frac{3}{8}(4 - x_{1}) = \frac{3}{8}(4 - 1.5) = \frac{15}{16} = 0.9375, x_{4} = 2 - x_{1} - x_{2} = 2 - 1.5 - 0.9375 = -0.4375$:
\begin{equation}
\mathbf{x} = \left[
	\begin{array}{c}
	1.5 \\ 0.9375 \\ 0 \\ -0.4375 \\ 0
	\end{array}
	\right]
\end{equation}
(reject) \\
\ \\
\ \\

Possible Basic solution \#9:
Let $x_{4} = x_{5} = 0$, and the set of equalities become:
\begin{eqnarray*}
x_{1} + \frac{8}{3}x_{2} + x_{3} = 4 \\
x_{1} + x_{2} = 2 \\
2x_{1} = 3 \\
\end{eqnarray*}
Therefore, for possible basic solution \#9, $x_{4} = x_{5} = 0, x_{1} = \frac{3}{2} = 1.5, x_{2} = 2 - 1.5 = 0.5, x_{3} = 4 - x_{1} - \frac{8}{3} \times x_{2} = 4 - 1.5 - \frac{8}{3} \times 0.5 = \frac{7}{6} \approx 1.1667$:
\begin{equation}
\mathbf{x} = \left[
	\begin{array}{c}
	1.5 \\ 0.5 \\ 1.1667 \\ 0 \\ 0
	\end{array}
	\right]
\end{equation}
(reject) \\

Infeasible solutions result, when the following pairs of variables are set to zero: $x_{1} = x_{5} = 0$. This is because $2\times x_{1} + x_{5} \neq 0$. \\

Hence, there is only one basic solution \#1, which contain integers and no negative numbers.


%%%%%%%%%%%%%%%%%%%%%%%%%%%%%%%%%%%%%%%%%%%
\section{Transforming Optimization Problem with Piecewise Linear Functions into Linear Programming Problem}
\label{sec:optimizationproblemtransformation}

The given mathematical optimization problem for a given class of piecewise linear functions is:
\begin{eqnarray*}
f({\mathbf{x}}) = {\rm maximum} (\mathbf{c}^{T}_{1}\mathbf{x} + d_{1}, \mathbf{c}^{T}_{2}\mathbf{x} + d_{2}, \dots, \mathbf{c}^{T}_{p}\mathbf{x} + d_{p})
\end{eqnarray*}

For this I can convert it into a linear model by separating the $x_{i}$ and $d_{i}$ components, $\forall i \in {1, 2, ..., n}$
programming \cite[\S3.3, pages 83--86]{Haftka1992}\cite[\S1, pages 5-8]{Malucelli2005}. Add all the $x_{i}$ components and place them into the new objective function with the initial displacement $d_{1}$. For each additive/displacement term $d_{i}$ that causes the model to be piecewise linear, multiply it by a slack variables and constraint them within a range $d_{i}y_{i1} \leq z_{i} \leq d_{i}y_{i2}$ That is, re-formulate the problem as:
\begin{eqnarray*}
f({\mathbf{x}}) = {\rm maximum} (d_{1}y_{1} + \mathbf{c}^{T}_{1}\mathbf{x} + \mathbf{c}^{T}_{2}\mathbf{x} + \dots + \mathbf{c}^{T}_{p}\mathbf{x}) \\
{\rm subject\ to:} \\
%d_{2}y_{1a} \leq z_{1} \leq d_{1}y_{1b} \\
d_{2}y_{2a} \leq z_{2} \leq d_{2}y_{2b} \\
d_{3}y_{3a} \leq z_{3} \leq d_{3}y_{3b} \\
\dots \\
d_{p}y_{pa} \leq z_{p} \leq d_{p}y_{pb} \\
\end{eqnarray*}


Subsequently, I can convert the maximization problem in linear programming into a minimization problem if linear programming in the standard form \cite[\S3.4, page 48, Example 1]{Luenberger2008}. Negate ``the objective function'' to convert it to a minimization problem. For each constraint, add a unique slack variable to turn them into equalities in the standard form of linear programming.

%%%%%%%%%%%%%%%%%%%%%%%%%%%%%%%%%%%%%%%%%%%
\section{Simplex Method}
\label{ssec:simplexmethod}



%%%%%%%%%%%%%%%%%%%%%%%%%%%%%%%%%%%%%%%%%%%
\subsection{Simplex Method, Implemented Using Linear Algebra}
\label{ssec:simplexmethodlinearalgebra}

\begin{eqnarray*}
{\rm maximize}\ (-x_{1} + x_{2}) \\
{\rm subject\ to} \\
x_{1} - x_{2} \leq 2 \\
x_{1} + x_{2} \leq 6 \\
x_{1} \geq 0, x_{1} \geq 0 \\
\end{eqnarray*}

Transform this to normal form by introducing slack variables $x_{3}$ and $x_{4}$, the objective/maximization function and constraints become: \\
\begin{eqnarray*}
z - x_{1} + x_{2} = 0 \\
x_{1} - x_{2} + x_{3} = 2 \\
x_{1} + x_{2} + x_{4 } = 6 \\
{\rm where}\ x_{1} \geq 0, x_{1} \geq 0 \\
\end{eqnarray*}


In the augmented matrix form (of $\mathbf{Ax = b}$), it becomes:
\[  \mathbf{T_{0}} =  
\begin{array}{c}
	\begin{array}{cccccc}
	z & x_{1} & x_{2} & x_{3} & x_{4} & b \\
	\end{array}
	\\
	\left[ \begin{array}{cccccc}
	1 & -1 & 1 & 0 & 0 & 0 \\
	0 & 1 & -1 & 1 & 0 & 2  \\
	0 & 1 & 1 & 0 & 1 & 6
	\end{array} \right]
\end{array}
\]

Select as the column of the pivot the first column with a negative entry in Row 1. This is column 3, because of the $-1$.\\

Selection of the row of the pivot. Divide the right sides (2 and 6) by the corresponding entries of the selected column ($2 \div -1 = -2$ and $6/1 = 6$). The smaller quotient of the rightmost column is the second row, with the quotient of $-2$.\\

\begin{equation*}
\mathbf{T_{1}} =  
\begin{array}{c}
	\begin{array}{cccccc}
	z & x_{1} & x_{2} & x_{3} & x_{4} & b \\
	\end{array}
	\\
	\left[ \begin{array}{cccccc}
	1 & -1 & 1 & 0 & 0 & 0 \\
	0 & 1 & -1 & 1 & 0 & -2  \\
	0 & 1 & 1 & 0 & 1 & 6
	\end{array} \right]
\end{array}
\end{equation*}

Elimination by row operations. Replace Row 1 with the addition of Row 1 and Row 3.

\begin{equation*}
\mathbf{T_{2}} =  
\begin{array}{c}
	\begin{array}{cccccc}
	z & x_{1} & x_{2} & x_{3} & x_{4} & b \\
	\end{array}
	\\
	\left[ \begin{array}{cccccc}
	1 & 0 & 2 & 0 & 1 & 6\ {\rm (Row\ 1 + Row\ 3)} \\
	0 & 1 & -1 & 1 & 0 & -1  \\
	0 & 1 & 1 & 0 & 1 & 6
	\end{array} \right]
\end{array}
\end{equation*}

Since there are no more negative entries in Row 1 of $T_{2}$, the Simplex method terminates with an optimum value of 6, which is the last column entry in Row 1. \\

That is, the optimum solution to the linear programming problem is 6.

%%%%%%%%%%%%%%%%%%%%%%%%%%%%%%%%%%%%%%%%%%%
\subsection{Simplex Method, Graphical Representation}
\label{ssec:simplexmethodgraphicalrepresentation}

Fig. \ref{fig:q3b1} shows the solution space of the linear programming problem. Fig. \ref{fig:q3b2} shows the inital solution indicated with a small circle at (0, 0), and Fig. \ref{fig:q3b3} shows the final solution indicated with a small circle at (0, 6).

\begin{figure}
\centering 
\includegraphics[width=6in]{./q3b_fig/q3b1}
\caption{Solution space of the linear programming problem}
\label{fig:q3b1}
\end{figure}

\begin{figure}
\centering 
\includegraphics[width=6in]{./q3b_fig/q3b2}
\caption{Solution space of the linear programming problem, with the inital solution indicated with a small circle at (0, 0).}
\label{fig:q3b2}
\end{figure}

\begin{figure}
\centering 
\includegraphics[width=6in]{./q3b_fig/q3b3}
\caption{Solution space of the linear programming problem, with the final solution indicated with a small circle at (0, 6).}
\label{fig:q3b3}
\end{figure}





%%%%%%%%%%%%%%%%%%%%%%%%%%%%%%%%%%%%%%%%%%%%%
{\linespread{1}
\bibliographystyle{plain}
\bibliography{/data/research/antipastobibtex/references}
}
%%%%%%%%%%%%%%%%%%%%%%%%%%%%%%%%%%%%%%%%%%%%%
\end{document}